% Une ligne commentaire débute par le caractère « % »

\documentclass[a4paper]{article}

% Options possibles : 10pt, 11pt, 12pt (taille de la fonte)
%                     oneside, twoside (recto simple, recto-verso)
%                     draft, final (stade de développement)

\usepackage[utf8]{inputenc}   % LaTeX, comprends les accents !
\usepackage[T1]{fontenc}      % Police contenant les caractères français
\usepackage[francais]{babel}
\usepackage{fullpage}
\usepackage{multicol}

\usepackage{blindtext}



\usepackage{graphicx}  % pour inclure des images

%\pagestyle{headings}        % Pour mettre des entêtes avec les titres
                              % des sections en haut de page

 \title{  TP1 : Les Bases d'Android\\         % Les paramètres du titre : titre, auteur, date
  Programmation mobile}
\author{Mohamad Satea Almallouhi - Tony Nguyen\\
  \emph{M1 Génie Logiciel}\\
  Faculté des Sciences\\
Université de Montpellier.}
\date{27 Février 2024}



\begin{document}
\maketitle                    % Faire un titre utilisant les données
                              % passées à \title, \author et \date

%\begin{center}               % pour centrer
 % \includegraphics[scale=1]{}   % insertion d'une image
%\end{center}

% \begin{abstract}     % Résumé du travail

%   \emph{Description très succinte du problème et des différentes étapes de réalisation}

% \end{abstract}
\newpage
%\dominitoc  % initializer les minitoc
\tableofcontents
\newpage

\begin{multicols}{2}
  [
    Faire une vidéo, rapport+read.md(instruction) screenchot résultats + code. +bonus bien fait,beau,tests,Kotlin,latex
  \section{Hello world}
  Nous allons voir comment afficher du texte à l'écran dans une activité sa vue associé.
  ]
  Tout d'abord, dans un fichier xml, nous déclarons une balise \textbf{<TextView>} avec un attribut \textbf{android:text} avec la valeur Hello World!
  Afin de l'afficher à l'écran on va utilser ce \textbf{layout} dans une activité à l'aide de la fonction \textbf{setContentView()}.
  Il est également nécessaire d'indiquer dans le manifeste l'activité à lancer en entrée. Pour cela, dans le manifest, nous ajoutons la balise \textbf{<intent-filter>}

  
\end{multicols}
\section{Simple formulaire}
\begin{multicols}{2}
  \includegraphics{cuteGirl.jpeg}
  \blindtext
\end{multicols}
\begin{multicols}{2}
  [ 
    Qu'est ce qu'un "Intent" ? 
  ]
  \subsection{Internationalisation}
  Utilisation de ressources R
  \blindtext
  \subsection{Évenements}
  Utilisation du concept de ressources. Dans res/values/strings.xml, nous déclarons les différents valeurs, mais avec le même attribut name.
  \blindtext
  \subsection{Intent explicite}
  
  \blindtext
  \subsection{Intent implcite}
  \blindtext
  \section{Consultation les horaires de trains}
  \blindtext
  \section{Simple d’agenda}
  \blindtext
\end{multicols}
\end{document}